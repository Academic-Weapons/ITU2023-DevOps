\documentclass{article}
\usepackage{amsmath}
\usepackage{graphicx}
\usepackage{hyperref}
\usepackage[inkscapeformat=png]{svg}

% Use a sans-serif font
\renewcommand{\familydefault}{\sfdefault}

% Adjust margins
\usepackage[margin=1.2in]{geometry}

\begin{document}

\begin{titlepage}
    \centering
    \includesvg[width=300px]{../images/itu_logo.svg}\\
    \vspace*{\fill}
    {\LARGE \textbf{DevOps Weapons - Final Report} \par}
    \vspace{1cm}
    {\large \today \par}
    \vspace{1cm}
    {\large Author 1 \\ Author 2 \\ Author 3 \par}
    \vspace*{\fill}
\end{titlepage}

% Table of contents
\tableofcontents
\newpage

\section{System Perspective}
Lorem ipsum dolor Lorem ipsum dolor Lorem ipsum dolor Lorem ipsum dolor Lorem ipsum dolor Lorem ipsum dolor Lorem ipsum dolor Lorem ipsum dolor Lorem ipsum dolor Lorem ipsum dolor Lorem ipsum dolor Lorem ipsum dolor Lorem ipsum dolor Lorem ipsum dolor Lorem ipsum dolor Lorem ipsum dolor Lorem ipsum dolor Lorem ipsum dolor Lorem ipsum dolor Lorem ipsum dolor Lorem ipsum dolor Lorem ipsum dolor Lorem ipsum dolor Lorem ipsum dolor Lorem ipsum dolor Lorem ipsum dolor Lorem ipsum dolor Lorem ipsum dolor Lorem ipsum dolor Lorem ipsum dolor Lorem ipsum dolor Lorem ipsum dolor Lorem ipsum dolor Lorem ipsum dolor Lorem ipsum dolor Lorem ipsum dolor Lorem ipsum dolor Lorem ipsum dolor Lorem ipsum dolor Lorem ipsum dolor Lorem ipsum dolor Lorem ipsum dolor Lorem ipsum dolor Lorem ipsum dolor Lorem ipsum dolor Lorem ipsum dolor Lorem ipsum dolor Lorem ipsum dolor Lorem ipsum dolor Lorem ipsum dolor Lorem ipsum dolor Lorem ipsum dolor Lorem ipsum dolor Lorem ipsum dolor Lorem ipsum dolor Lorem ipsum dolor Lorem ipsum dolor Lorem ipsum dolor Lorem ipsum dolor Lorem ipsum dolor Lorem ipsum dolor Lorem ipsum dolor Lorem ipsum dolor Lorem ipsum dolor Lorem ipsum dolor Lorem ipsum dolor Lorem ipsum dolor Lorem ipsum dolor Lorem ipsum dolor Lorem ipsum dolor Lorem ipsum dolor Lorem ipsum dolor Lorem ipsum dolor Lorem ipsum dolor Lorem ipsum dolor Lorem ipsum dolor Lorem ipsum dolor Lorem ipsum dolor Lorem ipsum dolor Lorem ipsum dolor Lorem ipsum dolor Lorem ipsum dolor Lorem ipsum dolor Lorem ipsum dolor 

\section{Process Perspective}

\subsection{\Large Security analysis}
\subsubsection{\Large Risk Identification}

{\large Identify Assets}

\begin{itemize}
    \item Web application
    \item Dockerhub
    \item GitHub
    \item Digital Ocean
    \item Grafana / Prometheus
    \item SonarCube
    \item Teams
\end{itemize}

\noindent {\large Identify Threat Sources}

\begin{itemize}
    \item SQL injection (very relevant)
    \item Password leakage
    \item GitHub is public (avoid committing sensitive information)
    \item Unused ports are closed off / behind capable firewalls
    \item Admin passwords and their strength on all relevant platforms
    \item Vulnerability of used platforms / frameworks
    \item Malicious code / images on Dockerhub
    \item Individual passwords for GitHub
    \item Teams channel is full of sensitive things
    \item Access to personal developer computers
\end{itemize}

\noindent {\large Construct Risk Scenarios}

An attacker could attempt a SQL injection with the goal of downloading the database, or parts hereof. This would expose the stored user information -- unhashed email addresses and encrypted passwords -- potentially harming users.

Suppose a developer commits sensitive information -- such as passwords or ssh keys -- to the public GitHub repository. An attacker then uses this information for their malicious intents, which would be almost limitless with root access.

A developers personal computer is breached because they have low strength passwords, allowing attackers immediate access to server, GitHub, DockerHub, pushing malicious code, and many other things.

\subsubsection{\Large Risk Analysis}

{\large Determine Likelihood, Impact & Potential discussions}

\begin{itemize}
    \item SQL Injection
    \begin{itemize}
        \item As we utilize the prepared statements found in the Java interface SQL, the likelihood is almost entirely bound by the risk of vulnerabilities in this interface. Outside of this, the risk of SQL injections are extremely low, as we only ever query with the above statements.
        \item The impact would be moderate. The attackers would be able to retrieve emails and hashed passwords. However, since the passwords are salted and hashed, the password leak would not be detrimental to users. The leaking of emails is a privacy concern.
        \item Should this happen, users must be informed that their emails were potentially leaked, and that their passwords should still be save, as we have stored them responsibly (in that we don't store them).
    \end{itemize}
    \item Password leakage
    \begin{itemize}
        \item As touched upon above, we salt and hash the passwords, and do not store any actual passwords. The risk here is the same a above.
        \item Should the stored password hash be leaked, it would be of negligible impact.
    \end{itemize}
    \item Public GitHub
    \begin{itemize}
        \item As this is dependent upon the developers, who are all currently humans, the likelihood is possible bordering on likely.
        \item The impact of such a mistake, can in worst case scenarios be severe. A private key being available on a public repository is about as severe as possible.
        \item As this scenarios actually happened to us. Early on, a private ssh key was pushed to the repository. As a response to this, the key should be changed immediately, and talks with developers, concerning how secrets are handled, should take place as soon as possible. Depending on the severity of what was leaked, a full code review should take place, to identify possible newly introduced vulnerabilities.
    \end{itemize}
    \item Port & Firewalls
    \begin{itemize}
        \item The likelihood a port is mistakenly open or a hole in the firewall exists is possible.
        \item The impact is dependent on what is exposed said port, or said hole in the firewall. The range of this, is the entire spectrum of impact severity.
        \item Once a hole or unintended port is found, both should be closed, and depending on what was exposed, a review of malicious changes should happen.
    \end{itemize}
    \item Shared admin password being leaked/cracked
    \begin{itemize}
        \item As the admin passwords have been shared on the teams channel, the likelihood is dependent on many factors, however, so guesstimate it here, we put it as possible. It could potentially be "likely" due to the many different attack surfaces that could expose it; teams accounts, personal computers, and the password itself.
        \item Severe++.
        \item There should not be a single root password. Instead, developers that need to have admin access, should have their own user that has 2 factor authentication as well as a significantly strong password, with the necessary permissions. Should a developer's user be breached, then the individual users make it simple to identify what changes the breached user made, if any, as well as easily and quickly restrict its access while maintaining the other developers'.
    \end{itemize}
    \item Vulnerability in used platforms / frameworks
    \begin{itemize}
        \item The likelihood here is arguably likely, if not very likely, simply due to the amount of dependencies a project like this relies on.
        \item The impact depends on what platform and or framework has a vulnerability, and how the project uses it.
        \item In reality, the likelihood can be kept nearer unlikely by staying up to date with all patches and updates to said dependencies.
    \end{itemize}
    \item Image tampering on Dockerhub
    \begin{itemize}
        \item The likelihood of this depends on the likelihood of an attacker getting our Dockerhub credentials, and/or an adversary is capable of acting as a man-in-the-middle and intercepting our messages to Dockerhub and replacing messages with malicious images. For the first scenario, see the admin password section, and for the latter, very unlikely as the communication between us and Dockerhub is encrypted with TLS.
        \item Severe, as it allows arbitrary code execution.
    \end{itemize}
    \item Developer passwords and their strengths
    \begin{itemize}
        \item As we currently have no policy for neither password strength nor two factor authentication, the likelihood must be at least likely.
        \item Severe, as it gives essentially arbitrary code execution.
        \item All on the project should use two factor authentication as well as a strong password.
    \end{itemize}
    \item Teams channel
    \begin{itemize}
        \item As the teams channel is accessible with each developers Microsoft account, and as mentioned above we have no policy for securing such credentials, it's also at the very least likely.
        \item Since some secrets have been shared through this channel, the impact could be severe.
        \item Don't share secrets in a teams channel. It's a teams channel...
    \end{itemize}
    \item Personal developer computers and the risks associated.
    \begin{itemize}
        \item Unlikely. The likelihood is directly related to how careful the developers are with their computer. However, it also requires physical access, as well as breaking into the locked machine. This does not necessarily make it more difficult or unlikely, but it is a significant hurdle, unless strongly motivated as it also carries a higher risk for the attacker.
        \item Should all of the above succeed for the attacker, the impact cannot be less than severe.
    \end{itemize}
\end{itemize}

\subsubsection{\Large Pen-testing}

Using the OWASP ZAP program, we were able to identify a short list of possible security issues with our own application. Of this list, we decided to begin by fixing the Content Security Policy header. In our application, we had not at this point made any CSP, and, therefore, a vulnerability to cross site scripting (XSS) and data injection attacks, to list a few.

As our application uses the Spring framework, it felt natural to use the Security tools Spring provides. In the class ContentSecurityPolicyConfiguration.java, we dictate our CSP. The actual policy used, is a policy found online that should cover the most basic issues.

The goal with setting this policy, is to limit the things that a users' browser will execute. For example, we only allow image sources to come from *.gravatar.com, which is the website we get user avatars from. Any other image won't be load by the browser. The interesting thing with CSP's is, we are not limiting what the application will do, but instead limiting what the users' browser will do. This is why it's a useful tool against XSS, as we can limit what scrips can and cannot be executed.

\section{Lessons Learned}
Lorem ipsum dolor Lorem ipsum dolor Lorem ipsum dolor Lorem ipsum dolor Lorem ipsum dolor Lorem ipsum dolor Lorem ipsum dolor Lorem ipsum dolor Lorem ipsum dolor Lorem ipsum dolor Lorem ipsum dolor Lorem ipsum dolor Lorem ipsum dolor Lorem ipsum dolor Lorem ipsum dolor Lorem ipsum dolor Lorem ipsum dolor Lorem ipsum dolor Lorem ipsum dolor Lorem ipsum dolor Lorem ipsum dolor 


%\bibliographystyle{plain}
%\bibliography{references}

\end{document}