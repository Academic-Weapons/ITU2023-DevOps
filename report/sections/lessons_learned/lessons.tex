\section{Lessons Learned}

\subsection{Onboarding new members}
One of the major challenges the team faced, was the addition of two new developers weeks into the course. 4 developers had worked on the refactoring of the MiniTwit we took over, and was almost completely done. The two new developers had little to no knowledge of Spring Boot as a framework. Being integrated into a team, where a lot of the "code" work was already done, made for a steep learning curve. The existing developers made themselves available to explain and help with questions. A small "enhancement" task was also created for the new-comers to finish, allowing them to try their hand at Spring Boot, GitHub Issues and the pipeline. This task was a great way to bring in new people who hadn't worked with the technologies before. 

\subsection{Scheduling}
Another issue the team experienced, was the disjoint schedules. The developers did not all have the same schedule, meaning working as pairs or in a group setting was only possible every once in a while. This issue was mostly overcome by adopting a sort of "remote work" attitude. Developers were expected to contribute to the project equally, however, when these contributions were to be made, was up to each developer.

\subsection{Missing features, importance of planning}
For week 05, one of the features we were supposed to add was an ORM framework (or similar database layer abstraction). We discussed internally, and chose to use Spring Data JPA. However, we never added it as a GitHub Issue as we normally would, and due to a miscommunication, no group member actually picked up the task - therefore it was never implemented. This highlights the need for proper communication and tracking of issues, in an environment where we rarely had the time to meet up physically. 

\subsection{Access to different platforms}
One of the main challenges we encountered during our project was ensuring everyone had access to the various platforms and technologies we were using. Different websites have different ways of verifying who you are, which added to the complexity.

For instance, with DigitalOcean, only one person could access our servers at first. This wasn't very convenient, so we set up a team within the platform, which allowed the entire group to have access. When we incorporated the ELK stack into our system, we needed to create another team and give everyone access to that as well.

Additionally, monitoring and logging each required separate passwords. Figuring out how to securely and efficiently manage all these access points and passwords was challenging.

\subsection{Implementing infrastructure as code}
We initially planned to use "infrastructure as code" through Vagrant to set up droplets for our project. However, without a complete understanding of our project's requirements - such as what it needed to run, and how we were going to launch it - this approach became overwhelming.

Because of this, we decided to create the droplets manually. This meant logging into each droplet via SSH and performing the necessary steps to get our application running, thereby postponing our plans to implement infrastructure as code.

However, this decision later posed a problem when we wanted to use Terraform, another tool for infrastructure as code. We found it difficult to backtrack exactly what had been done on each droplet during our manual setup. We have managed to configure Terraform to interact with DigitalOcean to create droplets as required and run Docker images as needed. 

If we had used Terraform from the start, it might have made things easier. However, it's likely we would have encountered similar issues as we did when using Vagrant. The probable solution to this issue is to have a comprehensive understanding of the project requirements before we start.


\section{Reflection}
% \textbf{TODO} Also reflect and describe what was the "DevOps" style of your work. For example, what did you do differently from previous development projects and how did it work?

From the very beginning, we started implementing good practices of work, such as an agile methodology, feedback loop, strict commit policies, and fair distribution of a workload. Since we knew each other strengths - we could focus on what we are good at, while also providing good learning possibilities for those interested in developing new skills. 

"DevOps" approach we took, was quite different from what we worked with before - it was more structured, better organized, and thought ahead. Also for some of us, it was our first hands-on experience of creating a system from the bottom up. We found that working in a DevOps style improved our efficiency, productivity, and satisfaction. We were able to deliver high-quality product faster, more reliably, and with much more enjoyment, while also learning new skills and technologies.

% We write in 3.2 that we didnt have same schedules so we can do standups, missed it. told u i went wild xd

We established a well-working feedback loop between team members more focused on the code, and the ones with a priority set towards deployment. 

As we had a discussion among ourselves, about what we liked the most about the course and our performance, we can share a team insight here as a reflection. One of the most mentioned things was a monitoring and logging system (Prometheus and Grafana) to collect and visualize data about the performance and health of our software and infrastructure. This helped us to identify and troubleshoot issues and surprised us with how well it did the job. The other thing was a smooth and faultless pipeline, which caused some trouble at the beginning, but has proven to be an irreplaceable asset.


\section{Conclusion}
In this project, we gained significant experience in various technical skills and DevOps methodologies. This allowed us to create a functional and reliable application, while also reducing the risks and costs associated with software development and operations. We worked together, sharing a common goal that kept us focused and motivated throughout.

Despite encountering challenges such as integrating new members mid-project and coordinating different schedules, we managed. This underscored the importance of flexibility and communication in a team setting.

We became adept in utilizing tools such as Spring Boot, Docker Swarm, ELK Stack, Prometheus, and Grafana, which enriched our technical skills. Our adoption of a DevOps approach was key to streamlining our development pipeline, reinforcing its effectiveness in software development.

In summary, we consider this project a success and a highly valuable learning experience.  


\section{Appendix}
Our groups Github can be found here, \href{https://github.com/Academic-Weapons/ITU2023-DevOps}{https://github.com/Academic-Weapons/ITU2023-DevOps}. All the corresponding links to monitoring, logging and more can be found in the readme file. 